\documentclass[12pt]{article}

% --- Packages ---
\usepackage{amsmath, amssymb, amsthm}
\usepackage{mathtools}
\usepackage{enumitem}
\usepackage{geometry}
\usepackage{algorithmic} % Use algorithmic package for the algorithm block
% Defines a new environment named 'theorem' that prints "Theorem" and is numbered by section.
\newtheorem{theorem}{Theorem}[] 

% Defines a new environment named 'lemma' that prints "Lemma" and shares the numbering counter with 'theorem'.
\newtheorem{lemma}[theorem]{Lemma} 

% Defines an unnumbered environment named 'remark' that prints "Remark".
\newtheorem*{remark}{Remark} 
% --- Custom Macros ---
\newcommand{\getsr}{\stackrel{\$}{\leftarrow}}
\newcommand{\Enc}{\text{Enc}}
\newcommand{\Dec}{\text{Dec}}
\newcommand{\Adv}{\text{Adv}}
\newcommand{\calO}{\mathcal{O}}

\title{Chapter 1 - Real and Complex Number Systems}
\date{}
\geometry{margin=1in}
\setlength{\parindent}{0pt}
\begin{document}
\maketitle{}
\section*{Ordered Sets}
\subsection*{Quick Definitions}
\begin{itemize}
    \item A order on set S is a relation denoted by $<$ with following properties
    \begin{itemize}
        \item if $x, y \in S$, only one of following statements are true: $x < y, x = y, y < x$
        \item If $x, y, z \in S$ and $x < y, y< z$ then $x < z$
    \end{itemize}
    \item Suppose S is an ordered set and $E \subset S$ then if $\exists \beta \in S, \forall x \in E, x \leq \beta$. Then $E$ is bounded above by $\beta$
    \item Least upperbound property: $\alpha$ is upperbound of $E$ and if $\gamma < \beta$ $\gamma$ is not upperbound of $E$. Call $\alpha = \sup E$.
\end{itemize}

\subsection*{Proofs}
\begin{theorem}
Suppose S is an ordered set with LUB property, $B \subset S$ and B is not an empty set and B is bounded below. Let L be the set of all lower bounds of B, then $\alpha = \sup L = \inf B \in S$.
\end{theorem}
\begin{proof}
Let L be the set of all lowerbounds of $B$. Since B is bounded below, L is not empty. Then since S has LUB property, sup L $\in S$.

We now prove $\alpha \in L$. For all beta < alpha, notice beta is not an upperbound of L by definition of upperbounds. Then, beta is not in B because every element in B is an upperbound of L. Thus, we have $\beta >= \alpha, \forall \beta \in S$, then by definition of L, $\alpha in L$.
We now prove $\forall \beta < \alpha$, $\beta$ is not greatest lower bound because $\alpha > \beta$ is $\alpha$ is a lower bound of B.

Thus, $\alpha = \inf B$.
\end{proof}
\end{document}